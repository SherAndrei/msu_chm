\documentclass[12pt]{article}

%Russian-specific packages
%--------------------------------------
\usepackage[T2A]{fontenc}
\usepackage[utf8]{inputenc}
\usepackage[english, russian]{babel}
%--------------------------------------
%for search in russian
\usepackage{cmap}

%Math-specific packages
%--------------------------------------
\usepackage{amsmath}
\usepackage{amssymb}

%Format-specific packages
%--------------------------------------
\usepackage[left=2cm,
            right=2cm,
            top=2cm,
            bottom=2cm,
            bindingoffset=0cm]{geometry}
%--------------------------------------

\usepackage{amsthm}

\newtheorem*{definition}{Определение}
\newtheorem*{theorem}{Теорема}

%Roman enum items
\usepackage{enumerate}

\def\bigO{ \underline{\underline{\mathcal{O}}} }

\begin{document}
    \begin{center}
    Численное решение дифференциального уравнения второго порядка. Отчет.
    \end{center}
    \begin{center}
    Шерстобитов Андрей, задача 4
    \end{center}
    Общая постановка задачи:
    \[\begin{cases}
        -y''(x) +p(x)y(x) = f(x),\ p(x) \geq 0\\
        y(0) = 0 \\
        y'(1) = 0
    \end{cases}\]
    Предлагается исследовать разностную схему вида
    \[\begin{cases}
        -\frac{y_{k+1}-2y_k+y_{k-1}}{h^2}+p_ky_k = f_k,\ h = \frac{2}{2N-1},\ 1 \leq k \leq N-1 \\
        x_k = kh,\ f_k := f(x_k),\ p_k := p(x_k) \\
        y_0 = 0 \\
        y_N = y_{N-1}
    \end{cases}\]
    Запишем канонический вид. Найдем коэффициенты для краевых условий
    \begin{align*}
        k &= 1: \frac{y_2 - 2y_1}{h^2} + p_1y_1= f_1 \\
        k &= N-1: \frac{-y_{N-1} + y_{N-2}}{h^2} + p_{N-1} y_{N-1} = f_{N-1}
    \end{align*}
    Задача в матричном виде имеет вид:
    \[
    -\left(\begin{array}{cccccc}
        \frac{-2}{h^2}& \frac{1}{h^2}  &&&& 0 \\
        \frac{1}{h^2} & \frac{-2}{h^2} & \frac{1}{h^2} &&& \\
        &&\cdots&\cdots&& \\
        &&& \frac{1}{h^2} & \frac{-2}{h^2} & \frac{1}{h^2} \\
        0 &&&               & \frac{1}{h^2}  & \frac{-1}{h^2} \\
    \end{array}\right)
    \left(\begin{array}{c}
        y_{1}\\
        \\
        \vdots\\
        \\
        y_{N-1}\\
    \end{array}\right)
    +
    \left(\begin{array}{ccc}
        p_1  && 0 \\
        && \\
        &\ddots& \\
        && \\
        0 && p_{N-1} \\
    \end{array}\right)
    \left(\begin{array}{c}
        y_{1}\\
        \\
        \vdots\\
        \\
        y_{N-1}\\
    \end{array}\right)
    =
    \left(\begin{array}{c}
        f_{1}\\
        \\
        \vdots\\
        \\
        f_{N-1}\\
    \end{array}\right)
    \]
    Будем использовать сеточную интегральную норму
    \[\Vert y_h\Vert^2_h = (y_h,y_h)_h= \sum_{k=1}^{N-1}y^2_kh\]
    согласованной с нормой $\Vert y(x)\Vert^2_{L^2(0,1)} = \int_0^1y^2(x)dx$ исходной задачи.

    План:
    \begin{enumerate}[I.]
        \item Доказать, что разностная схема имеет порядок аппроксимации $O(h^2)$.
        \item Доказать устойчивость разностный схемы.
        \item Доказать, что есть сходимость $O(h^2)$.
        \item Выписать алгоритм решения для метода Фурье и метода прогонки.
        \item Показать, что сходимость численно действительно $O(h^2)$ либо иначе.
    \end{enumerate}

    \newpage

    \begin{enumerate}[I.]
        \item Докажем аппроксимацию второго порядка на решении:
        \begin{enumerate}
            \item $\Vert L_h(y)_{Y_h}-f_h\Vert_{F_h}\leq\bigO(h^2)$
            \[\max_{x_k}\left|-\frac{y(x_k+h)-2y(x_k)+y(x_k-h)}{h^2}+p(x_k)y(x_k)-f(x_k)\right|=\]
            \[y(x_k\pm h)=y(x_k)\pm hy'(x_k)+\frac{h^2}{2}y''(x_k)\pm\frac{h^3}{6}y'''(x_k)+\bigO(h^4)\]
            \[=\max_{x_k}\left|-\frac{h^2y''(x_k)+\bigO(h^4)}{h^2}+p(x_k)y(x_k)-f(x_k)\right|=\]
            \[=\max_{x_k}\left|-y''(x_k)+\bigO(h^2)+p(x_k)y(x_k)-f(x_k)\right|=\bigO(h^2)\]
            \item $\Vert l_h(y)_{Y_h}-\varphi_h\Vert_{\Phi_h}\leq\bigO(h^2)$
                \[y_0 = 0:\ \Vert y(0)-0\Vert_{\Phi_h}=0\]
                \begin{multline*}
                    y_N = y_{N-1}:\ \left\Vert y\left(1+\frac{h}{2}\right)-y\left(1-\frac{h}{2}\right)\right\Vert_{\Phi_h}= \\
                    y\left(1\pm \frac{h}{2}\right)=y(1)\pm \frac{h}{2}y'(1)+\frac{h^2}{2}y''(x_k)+\bigO(h^3) \\
                    =\left\Vert hy'(1)+\bigO(h^3)\right\Vert_{\Phi_h}=\bigO(h^3)
                \end{multline*}
            \item Условия нормировки
            \begin{align*}
                \lim_{h\rightarrow0}\Vert f_h-(f)_{F_h}\Vert_{F_h}&=0 \Rightarrow f(x_k)-f(f_k) = 0\\
                \lim_{h\rightarrow0}\Vert \varphi_h-(\varphi)_{\Phi_h}\Vert_{\Phi_h}&=0 \Rightarrow (0\ 0)^T - (0\ 0)^T = 0
            \end{align*}
        \end{enumerate}
        Значит схема имеет \textbf{второй порядок аппроксимации}.
        \newpage

        \item Напомним определение устойчивости разностной схемы.
        \begin{definition}
            Пусть уравнение $y''(x)=f(x)$ доопределено краевыми
            условиями на разных концах отрезка. Разностная схема
            $A_hy_h = f_h$ линейной задачи устойчива, если существуют $C$, $h_0$ такие, что для
            произвольных $A_hy^{(1,2)}_h = f^{(1,2)}_h$ выполняется оценка
            \[\Vert y^{(1)}_h-y^{(2)}_h\Vert_h\leq C\Vert f^{(1)}_h -f^{(2)}_h\Vert_h\ \forall h\leq h_0\]
            с константой $C$, не зависящей от $h$.
        \end{definition}

        Будем доказывать устойчивость разностной схемы энергетическим методом.
        Запишем нашу дифференциалную задачу
        \[-y''(x)+p(x)y(x)=f(x),\ y(0) = y'(1) = 0,\ p(x)\geq 0\]
        Умножим уравнение на $y(x)$, и результат проинтегрируем по отрезку $[0, 1]$
        \[\int_0^1 (-y''y+py^2)dx = \int_0^1fydx \]
        \[\int_0^1 -y''ydx+ \int_0^1py^2 dx = \int_0^1fydx \]
        Проинтегрируем по частям первое слагаемое
        \[\int_0^1 -y''ydx = \int_0^1-ydy' = -yy'\vert^1_0 - \int_0^1y'd(-y) = \int_0^1(y')^2dx\]
        Получили интегральное тождество
        \[\int_0^1 (y'(x))^2dx+ \int_0^1py^2 dx = \int_0^1fydx \]
        Оценим слева через неравенство, связывающее интегралы от квадратов
        функции и ее производной. Так как $y(0) = 0$, то справедливо следующее:
        \[y(x_0) = \int_0^{x_0}y'(x)dx\]
        Применим интегральную форму неравенства Коши-Буняковского:
        \[|y(x_0)|^2 = \left|\int_0^{x_0}y'dx\right|^2\leq\left(\int_0^{x_0}1^2dx\right)\left(\int_0^{x_0}(y')^2dx\right)\leq\int_0^{x_0}(y')^2dx\leq\int_0^{1}(y')^2dx\]
        После интегрирования по $x_0$ по отрезку $[0,1]$ обеих частей получим искомое равенство
        \[\int_0^1|y(x_0)|^2dx_0 \leq \int_0^{1}(y')^2dx\int_0^1dx_0 \Leftrightarrow \int_0^1y^2dx\leq\int_0^1(y')^2dx\]
        Оценку справа выведем из разности квадратов:
        \[0\leq\left(\int_0^1fdx-\int_0^1ydx\right)^2\leq\left(\int_0^1fdx\right)^2-2\int_0^1fydx+\left(\int_0^1ydx\right)^2\]
        \[\Rightarrow\int_0^1fydx\leq\frac{1}{2}\left(\int_0^1f^2dx + \int_0^1y^2dx\right)\]

        Таким образом имеем:
        \[\int_0^1y^2dx\leq\int_0^1 (y'(x))^2dx+ \int_0^1py^2 dx = \int_0^1fydx\leq\frac{1}{2}\left(\int_0^1f^2dx + \int_0^1y^2dx\right)\]
        Таким образом верна оценка
        \[\int_0^1y^2dx\leq\int_0^1f^2dx\Rightarrow\Vert y\Vert_{L_2(0,1)}\leq\Vert f\Vert_{L_2(0,1)}\]
        Это означает устойчивость дифференциальной задачи по правой части.

        Докажем теперь устойчивость разностной схемы.
        \[-\frac{y_{k+1}-2y_k+y_{k-1}}{h^2}+p_ky_k = f_k,\ 1 \leq k \leq N-1,\ y_0 = 0,\ y_N = y_{N-1}\]
        Умножим на $y_k$ и просуммируем от $1$ до $N-1$. Так как $y_0 = 0,\ y_N = y_{N-1}$
        \[-\frac{1}{h^2}\left(\sum_{k=1}^{N-1}\left(y_{k+1}-2y_k+y_{k-1}\right)y_k\right)=-\frac{1}{h^2}\left(\sum_{k=1}^{N-1}\left(y_{k+1}-y_k-y_k+y_{k-1}\right)y_k\right)=\]
        \[=-\frac{1}{h^2}\sum_{k=1}^{N-1}\left(y_{k+1}-y_k\right)y_k+\frac{1}{h^2}\sum_{k=1}^{N-1}\left(y_k-y_{k-1}\right)y_k=\]
        \[=-\frac{1}{h^2}\sum_{k=2}^{N}\left(y_{k}-y_{k-1}\right)y_{k-1}+\frac{1}{h^2}\sum_{k=1}^{N-1}\left(y_k-y_{k-1}\right)y_k=\frac{1}{h^2}\sum_{k=1}^{N}(y_k-y_{k-1})^2\]
        Получили конечномерный аналог интегрального тождества:
        \[\frac{1}{h^2}\sum_{k=1}^N(y_k-y_{k-1})^2+\sum_{k=1}^{N-1}p_ky_k^2=\sum_{k=1}^{N-1}f_ky_k\]
        Для оценки слева докажем сеточный аналог неравенства для функции и ее производной в точках $k=1,...,N-1$.
        Так как $y_0 = 0$, справедливо следующее:
        \[y_k=\sum_{i=1}^{k}(y_i-y_{i-1})\]
        Воспользуемся неравенством Коши-Буняковского и $y_N=y_{N-1}$
        \[y_k^2\leq\left(\sum_{i=1}^k1^2\right)\left(\sum_{i=1}^k(y_i-y_{i-1})^2\right)\leq (N-1)\sum_{i=1}^{N-1}(y_i-y_{i-1})^2\]
        Суммируя до $N-1$ обе части, при $h=\frac{2}{2N-1}$ получаем оценку:
        \[\sum_{k=1}^{N-1}y_k^2\leq(N-1)^2\sum_{k=1}^{N-1}(y_k-y_{k-1})^2\leq\frac{1}{h^2}\sum_{k=1}^{N-1}(y_k-y_{k-1})^2\]
        Найдем аналогично дифференциальному неравенству оценку справа
        \[0\leq\left(\sum_{k=1}^{N-1}f_k-\sum_{k=1}^{N-1}y_k\right)^2\leq\left(\sum_{k=1}^{N-1}f_k\right)^2-2\sum_{k=1}^{N-1}f_ky_k+\left(\sum_{k=1}^{N-1}y_k\right)^2\]
        \[\Rightarrow\sum_{k=1}^{N-1}f_ky_k\leq\frac{1}{2}\left(\sum_{k=1}^{N-1}f_k^2+\sum_{k=1}^{N-1}y_k^2\right)\]
        Итоговая оценка имеет вид
        \[\sum_{k=1}^{N-1}y_k^2\leq\frac{1}{h^2}\sum_{k=1}^{N-1}(y_k-y_{k-1})^2+\sum_{k=1}^{N-1}p_ky_k^2=\sum_{k=1}^{N-1}f_ky_k\leq\frac{1}{2}\left(\sum_{k=1}^{N-1}f_k^2+\sum_{k=1}^{N-1}y_k^2\right)\]
        Таким образом
        \[\sum_{k=1}^{N-1}y_k^2\leq\sum_{k=1}^{N-1}f_k^2\Rightarrow\sum_{k=1}^{N-1}y_k^2h\leq\sum_{k=1}^{N-1}f_k^2h\Rightarrow\Vert y_h\Vert^2_h\leq\Vert f_h\Vert^2_h\]
        То есть \textbf{разностная схема устойчива}.
        \newpage
        \item Докажем, что у схемы есть сходимость порядка $\bigO(h^2)$.
        \begin{theorem}[Филиппов А.Ф.]
            Пусть выполнены следюущие условия:
            \begin{enumerate}
                \item операторы $L$, $l$ и $L^h$, $l^h$ - линейные;
                \item решение дифференциальной задачи $\exists!$;
                \item разностная схема аппроксимирует на решении дифференциальную задачу с порядком $p$;
                \item разностная схема устойчива;
            \end{enumerate}
            Тогда решение разностной схемы сходится к решению дифференциальной задачи с порядком не ниже $p$
        \end{theorem}
        Посмотрим на наши результаты
        \begin{enumerate}
            \item операторы $L$, $l$ и $L^h$, $l^h$ - действительно линейные;
            \item решение дифференциальной задачи $\exists!$, так как по условию $y$ и $f$ хорошие гладкие функции.
            \item разностная схема аппроксимирует на решении дифференциальную задачу с порядком $2$;
            \item разностная схема действительно устойчива;
        \end{enumerate}
        Таким образом решение разностной схемы \textbf{сходится} к решению дифференциальной задачи \textbf{с порядком не ниже $2$}.

        \newpage

        \item Выпишем алгоритм решения для метода Фурье и метода прогонки.
        \begin{enumerate}
            \item Метод Фурье
            \[Ay=f,\ A\in \mathbb{R}^{N-1\times N-1}\]
            \[A=
            -\left(\begin{array}{cccccc}
                \frac{-2}{h^2}& \frac{1}{h^2}  &&&& 0 \\
                \frac{1}{h^2} & \frac{-2}{h^2} & \frac{1}{h^2} &&& \\
                &&\cdots&\cdots&& \\
                &&& \frac{1}{h^2} & \frac{-2}{h^2} & \frac{1}{h^2} \\
                0 &&&               & \frac{1}{h^2}  & \frac{-1}{h^2} \\
            \end{array}\right)
            +
            \left(\begin{array}{ccc}
                p_1  && 0 \\
                && \\
                &\ddots& \\
                && \\
                0 && p_{N-1} \\
            \end{array}\right)
            =\tilde{A} +pI\]
            \begin{enumerate}
                \item Для сходимости решения требуется $A=A^T\geq0$. Это возможно $\Leftrightarrow$ $p\equiv const\geq0$
                \item Так как матрица $A=A^T\geq0$, значит 
                \[\exists\{(e_i,\lambda_i)_{i=1,\ldots,N-1}|Ae_i=\lambda_ie_i\},\ (e_i,e_j)_h=\delta_{ij}\ \forall i,j \]
                \item Собственные числа и вектора для матрицы $\tilde{A}$ находятся аналитически
                \begin{align*}
                    \tilde{e_i}=&\sqrt{2}\sin\frac{\pi(2m-1)i}{2N-1}  &m=1,\ldots,N-1\\
                    \tilde{\lambda}=&\frac{4}{h^2}\sin^2\left(\frac{\pi(2m-1)}{2(2N-1)}\right) &i=1,\ldots,N-1
                \end{align*}
                При этом
                \[\tilde{A}\tilde{e_i}=\tilde{\lambda_i}\tilde{e_i}\]
                \[(A-pI)\tilde{e_i}=A\tilde{e_i}-p\tilde{e_i}=\tilde{\lambda_i}\tilde{e_i}\Leftrightarrow A\tilde{e_i}=(\tilde{\lambda_i}+p)\tilde{e_i}\]
                Таким образом собственные вектора $A$: $e_i = \tilde{e_i}$, собственные числа $\lambda_i=\tilde{\lambda_i}+p$
                \item Можем применить метод:
                \[Ay=f,\ y=\sum_{i=1}^{N-1}c_ie_i,\ f=\sum_{i=1}^{N-1}d_ie_i,\ d_i=(f,e_i)_h\]
                \[A\left(\sum_{i=1}^{N-1}c_ie_i\right)=\sum_{i=1}^{N-1}c_i\lambda_ie_i=\sum_{i=1}^{N-1}d_ie_i\Rightarrow c_i=\frac{d_i}{\lambda_i},\ \lambda_i\neq0\]
                \[y_k=\sum_{i=1}^{N-1}c_ie_i(k),\ k=1,\ldots,N\]
            \end{enumerate}

            \newpage

            \item Метод прогонки
            \[Ay=f,\ A\in \mathbb{R}^{N-1\times N-1}\]
            \[A=
            \left(\begin{array}{cccccc}
                \frac{2}{h^2}+p_1& -\frac{1}{h^2}  &&&& 0 \\
                -\frac{1}{h^2} & \frac{2}{h^2}+p_2& -\frac{1}{h^2} &&& \\
                &&\cdots&\cdots&& \\
                &&& -\frac{1}{h^2} & \frac{2}{h^2}+p_{N-2} & -\frac{1}{h^2} \\
                0 &&&               & -\frac{1}{h^2}  & \frac{1}{h^2}+p_{N-1} \\
            \end{array}\right)\]
            \[
            y=\left(\begin{array}{c}
                y_{1}\\
                \\
                \vdots\\
                \\
                y_{N-1}\\
            \end{array}\right)\
            f=
            \left(\begin{array}{c}
                f_{1}\\
                \\
                \vdots\\
                \\
                f_{N-1}\\
            \end{array}\right)
            \]

            Перепишем матрицу $A$ следующим образом:
            \[\left(\begin{array}{cccccc}
                c_1& -b_1  &&&& 0 \\
                -a_2 & c_2& -b_2 &&& \\
                &&\cdots&\cdots&& \\
                &&& -a_{N-2} & c_{N-2} & -b_{N-2} \\
                0 &&& & -a_{N-1} & c_{N-1} \\
            \end{array}\right)\]
            Тогда задачу можно переписать следующем образом
            \[
            \begin{array}{cc}
                c_1y_1-b_1y_2=f_1, &k=1\\
                -a_ky_{k-1}+c_ky_k-b_ky_{k+1}=f_k &k=2,\ldots,N-2\\
                -a_{N-1}y_{N-2}+c_{N-1}y_{N-1}=f_{N-1},  &k=N-1
            \end{array}
            \]
            Перепишем первое уравнение
            \[ c_1y_1-b_1y_2=f_1 \Rightarrow y_1-\frac{b_1}{c_1}y_2=\frac{f_1}{c_1}\Rightarrow y_1=\alpha_2y_2+\beta_2,\ \alpha_2=\frac{b_1}{c_1},\ \beta_2=\frac{f_1}{c_1}\]
            Найдем $\alpha_{k+1}$ и $\beta_{k+1}$, используя полученную формулу $y_{k-1}=\alpha_ky_k+\beta_k$, подставив во второе уравнение
            \[-a_k(\alpha_ky_k+\beta_k)+c_ky_k-b_ky_{k+1}=(-a_k\alpha_k+c_k)y_k-\alpha_k\beta_k-b_ky_{k+1}=f_k\Rightarrow\]
            \[\Rightarrow(-\alpha_ka_k+c_k)y_k+(-b_k)y_{k+1}=\alpha_k\beta_k+f_k\Rightarrow y_k=\left(\frac{b_k}{c_k-\alpha_ka_k}\right)y_{k+1}+\frac{\alpha_k\beta_k+f_k}{c_k-\alpha_ka_k}\]
            \[\alpha_{k+1}=\frac{b_k}{c_k-\alpha_ka_k},\ \beta_{k+1}=\frac{\alpha_k\beta_k+f_k}{c_k-\alpha_ka_k}\]
            Рассмотрим последнее равенство, подставим в него $y_{N-2}=\alpha_{N-1}y_{N-1}+\beta_{N-1}$
            \[-a_{N-1}(\alpha_{N-1}y_{N-1}+\beta_{N-1})+c_{N-1}y_{N-1}=f_{N-1}\]
            \[y_{N-1}=\frac{f_{N-1}+\alpha_{N-1}\beta{N-1}}{c_{N-1}-a_{N-1}\alpha_{N-1}}\]
            Получили формулы для правой прогонки.
        \end{enumerate}

        \newpage

        \item Результаты численных подсчетов реализованных методов Фурье и прогонки.
    \end{enumerate}
\end{document}
