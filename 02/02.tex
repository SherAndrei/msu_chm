\documentclass[12pt]{article}

%Russian-specific packages
%--------------------------------------
\usepackage[T2A]{fontenc}
\usepackage[utf8]{inputenc}
\usepackage[english, russian]{babel}
%for search in russian
% \usepackage{cmap}
%--------------------------------------

%Math-specific packages
%--------------------------------------
\usepackage{amsmath}
\usepackage{amssymb}

%Format-specific packages
%--------------------------------------
\usepackage[left=2cm,
            right=2cm,
            top=2cm,
            bottom=2cm,
            bindingoffset=0cm]{geometry}
%--------------------------------------

\def\bigO{ \underline{\underline{\mathcal{O}}} }

\begin{document}
\begin{center}
Отчет по численному исследованию базовых разностных схем.
\end{center}
\begin{center}
Шерстобитов Андрей, задача 02
\end{center}

Для построения приближенного решения задачи
\[
    y'(x) + Ay(x) = 0,\ y(0) = 1,\ x \in [0, 1],\ A > 0
\]

С известным точным решением $y(x) = e^{-Ax}$ найдем порядок аппроксимации и проверим $\alpha$-устойчивость схем:
\begin{enumerate}
    \item $\frac{y_{k+1} - y_{k}}{h} + Ay_{k} = 0, y_0 = 1$
    \begin{multline*}
        \left|y'(x_k)+Ay(x_k)-\frac{y(x_k+h)-y(x_k)}{h}-Ay(x_k)\right|=\\
        \left|y'(x_k)-\frac{hy'(x_k)+\frac{h^2}{2}y''(\xi)}{h}\right| = \left|\frac{h}{2}y''(\xi)\right|=\bigO(h^1)
    \end{multline*}
    Разностная схема имеет первый порядок аппроксимации.
    \[\frac{y_{k+1}-y_k}{h}=0\Rightarrow P(\mu)=\mu-1\Rightarrow \mu=1\]
    Схема $\alpha$-устойчива
    \item $\frac{y_{k+1} - y_{k}}{h} + Ay_{k+1} = 0, y_0 = 1$
    \begin{multline*}
        \left|y'(x_k)+Ay(x_k)-\frac{y(x_k+h)-y(x_k)}{h}-Ay(x_k+h)\right|=\\
        \left|y'(x_k)-\frac{hy'(x_k)+\frac{h^2}{2}y''(\xi)}{h}-Ay'(\eta)h\right| = \left|\frac{h}{2}y''(\xi)-Ay'(\eta)h\right|=\bigO(h^1)
    \end{multline*}
    Разностная схема имеет первый порядок аппроксимации. \\
    Схема $\alpha$-устойчива (см. схему 1).
    \item $\frac{y_{k+1} - y_{k}}{h} + A\frac{y_{k+1} + y_{k}}{2} = 0, y_0 = 1$
    \begin{multline*}
        \left|y'(x_k)+Ay(x_k)-\frac{y(x_k+h)-y(x_k)}{h}-A\frac{y(x_k+h) + y(x_k)}{2}\right|=\\
        \left|\frac{2h(y'(x_k)+Ay(x_k))-2(y(x_k+h)-y(x_k))-Ah(y(x_k+h) + y(x_k))}{2h}\right| =\\
        \left|\frac{2h(y'(x_k)+Ay(x_k))-2(y'(x_k)h+y''(x_k)\frac{h^2}{2}+\bigO(h^3))-Ah(2y(x_k) + y'(x_k)h+\bigO(h^2))}{2h}\right| =\\
        \left|\frac{-y''(x_k)h^2+\bigO(h^3)-Ay'(x_k)h^2+\bigO(h^3)}{2h}\right| = \left|\frac{-h\overbrace{(y''(x_k)+Ay'(x_k))}^{=(y'(x)+Ay(x))'=0}+\bigO(h^2)}{2}\right| = \bigO(h^2)
    \end{multline*}
    Разностная схема имеет второй порядок аппроксимации. \\
    Схема $\alpha$-устойчива (см. схему 1).
    \item $\frac{y_{k+1} - y_{k-1}}{2h} + Ay_k = 0, y_0 = 1, y_1 = 1 - Ah$
    \begin{enumerate}
        \item Проверим схему
        \begin{multline*}
            \left|y'(x_k)+Ay(x_k)-\frac{y(x_k+h)-y(x_k-h)}{2h}-Ay(x_k)\right|=\\
            \left\|y(x_k\pm h)=y(x_k)\pm y'(x_k)h + y''(x_k)\frac{h^2}{2}\pm y'''(x_k)\frac{h^3}{6}+\bigO(h^4)\right\| = \\
            \left|\frac{2hy'(x_k)-2hy'(x_k)+\bigO(h^3)}{2h}\right| = \bigO(h^2)
        \end{multline*}
        \item Проверим краевое условие
        \[|y(h)-y_1|=|\overbrace{y(0)}^{=1}+y'(0)h+\bigO(h^2)-1+Ah|=|h\overbrace{(y'(0)+Ay(0))}^{=0}+\bigO(h^2)|=\bigO(h^2)\]
    \end{enumerate}
    Разностная схема имеет второй порядок аппроксимации.
    \[\frac{y_{k+1}-y_{k-1}}{2h}=0\Rightarrow P(\mu)=\mu^2-1\Rightarrow \mu=\pm 1\]
    Схема $\alpha$-устойчива.
    \item $\frac{1.5y_{k} - 2y_{k-1} + 0.5y_{k-2}}{h}+Ay_k= 0, y_0 = 1, y_1 = 1 - Ah.$
    \begin{enumerate}
        \item Проверим аппроксимацию схемы
        \begin{multline*}
            \left|y'(x_k)-\frac{1.5y(x_k)-2y(x_k-h)+0.5y(x_k-2h)}{h}\right|=\\
            \left|y'(x_k)-\frac{-2\left[-hy'(x_k)+\frac{h^2}{2}y''(x_k)+\bigO(h^3)\right]+\frac{1}{2}\left[-2hy'(x_k)+2h^2y''(x_k)+\bigO(h^3)\right]}{h}\right|=\\
            \left|-2\left[\frac{h^2}{2}y''(x_k)+\bigO(h^2)\right]+\frac{1}{2}\left[2h^2y''(x_k)+\bigO(h^2)\right]\right|=\bigO(h^2)
        \end{multline*}
        \item Аппроксимация краевого условия $\bigO(h^2)$ (см. схему 4)
    \end{enumerate}
    Разностная схема имеет второй порядок аппроксимации.
    \[\frac{1.5y_k-2y_{k-1}+0.5y_{k-2}}{h}=0\Rightarrow P(\mu)=1.5\mu^2-2\mu+0.5\Rightarrow\mu_1=1;\mu_2=\frac{1}{3}\]
    Схема $\alpha$-устойчива.
    \newpage
    \item $\frac{-0.5y_{k+2}+2y_{k+1}-1.5y_k}{h} + Ay_k = 0, y_0 = 1, y_1 = 1 - Ah.$
    \begin{enumerate}
        \item Проверим аппроксимацию схемы
        \begin{multline*}
            \left|y'(x_k)-\frac{-0.5y(x_k+2h)+2y(x_k+h)-1.5y(x_k)}{h}\right|=\\
            \left|y'(x_k)-\frac{-\frac{1}{2}\left[2hy'(x_k)+2h^2y''(x_k)+\bigO(h^3)\right]+2\left[hy'(x_k)+\frac{h^2}{2}y''(x_k)+\bigO(h^3)\right]}{h}\right|=\\
            \left|y'(x_k)-\frac{-\left[hy'(x_k)+h^2y''(x_k)+\bigO(h^3)\right]+\left[2hy'(x_k)+h^2y''(x_k)+\bigO(h^3)\right]}{h}\right|=\\
            \left|y'(x_k)+\left[y'(x_k)+hy''(x_k)+\bigO(h^2)\right]-\left[2y'(x_k)+hy''(x_k)+\bigO(h^2)\right]\right|=\bigO(h^2)
        \end{multline*}
        \item Аппроксимация краевого условия $\bigO(h^2)$ (см. схему 4)
    \end{enumerate}
    Разностная схема имеет второй порядок аппроксимации.
    \[\frac{-0.5y_{k+2}+2y_{k+1}-1.5y_k}{h}=0\Rightarrow P(\mu)=-0.5\mu^2+2\mu-1.5\Rightarrow\mu_1=1;\mu_2=3\]
    Схема \textbf{не} $\alpha$-устойчива.
\end{enumerate}

\begin{center}
    Результаты программы
\end{center}

\begin{tabular}{c c c c c c c}
№ &      E1      &      E2       &       E3      &      E6      &  m & A \\
1 & 1.920100e-02 & 1.847100e-03  &  1.840164e-04 & 1.839504e-07 &  1 & 1 \\
1 & 0.000000e+00 & 1.920100e-02  &  1.847100e-03 & 1.839403e-06 &  1 & 10 \\
1 & 9.043821e+19 & 2.656140e+95  &  0.000000e+00 & 0.000000e+00 &  1 & 1000 \\
2 & 1.766385e-02 & 1.831771e-03  &  1.838631e-04 & 1.839699e-07 &  1 & 1 \\
2 & 1.321206e-01 & 1.766385e-02  &  1.831771e-03 & 1.839387e-06 &  1 & 10 \\
2 & 0.000000e+00 & 0.000000e+00  &  0.000000e+00 & 0.000000e+00 &  1 & 1000 \\
3 & 3.068988e-04 & 3.065695e-06  &  3.065658e-08 & 5.842771e-12 &  2 & 1 \\
3 & 3.454611e-02 & 3.068988e-04  &  3.065695e-06 & 4.302614e-12 &  2 & 10 \\
3 & 9.607843e-01 & 0.000000e+00  &  0.000000e+00 & 0.000000e+00 &  2 & 1000 \\
4 & 1.000000e-01 & 1.000000e-02  &  1.000000e-03 & 1.000000e-06 &  2 & 1 \\
4 & 0.000000e+00 & 5.375719e+01  &  5.505286e-01 & 1.000000e-05 &  2 & 10 \\
4 & 5.070070e+22 & 7.321840e+129 &  0.000000e+00 &      inf     &  2 & 1000 \\
5 & 1.839902e+02 & 4.460325e+25  &       inf     &      inf     &  2 & 1 \\
5 & 0.000000e+00 & 7.383011e+25  &       inf     &      inf     &  2 & 10 \\
5 & 5.019147e+04 & 1.936319e+30  &  0.000000e+00 &      inf     &  2 & 1000 \\
6 & 1.679148e+02 & 1.758994e+43  &       inf     &      inf     &  2 & 1 \\
6 & 0.000000e+00 & 2.455656e+46  &       inf     &      inf     &  2 & 10 \\
6 & 3.468003e+12 & 4.839576e+81  &  0.000000e+00 &      inf     &  2 & 1000
\end{tabular}

\end{document}
